
\documentclass[a4paper]{article}
\usepackage[utf8]{inputenc}
\title{Solving the Problem of Manual Post-Processing with Prompt Engineering}
\author{}

\begin{document}

\maketitle

\begin{abstract}
This paper investigates the application of prompt engineering to reduce manual post-processing in natural language models. Our hypothesis is that prompt engineering, which involves providing the language models with pre-defined phrases or statements, can improve the output of the model and reduce the need for manual post-processing. We verify this hypothesis by collecting a set of training data and topics, setting up a baseline accuracy score for the same model without prompt engineering, and then feeding the same data and topics to the model with prompt engineering. The accuracy of the output from the LLM with prompt engineering is then compared to the baseline accuracy score to measure the improvement made. The results of our analysis verify that prompt engineering improves the accuracy of natural language models and reduces the need for manual post-processing.
\end{abstract}

\section{Introduction}
Natural language models (LLMs) are becoming increasingly popular for applications such as automated customer service,