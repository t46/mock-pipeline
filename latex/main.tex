\documentclass[12pt]{article}
\usepackage[utf8]{inputenc}
\usepackage[english]{babel}
\usepackage{amsmath}
\usepackage{graphicx}
\usepackage{hyperref}
\usepackage{natbib}
\setlength{\parindent}{4em}
\setlength{\parskip}{1em}

\begin{document}

\begin{titlepage}
	\centering
	\vspace*{8cm}
	{\huge\bfseries Mitigating Irrelevant Output in GPT-4 Large Language Models through Specific Prompts\par}
	\vspace{2cm}
	{\Large\itshape <Your Name>\par}
	\vfill
	{\large \today\par}
\end{titlepage}

\section*{Abstract}

This paper presents an empirical study on the problem of irrelevant sentence outputs when engaging with Large Language Models, with focus on GPT-4. As these models are instructed via prompts, they may return additional sentences along with the desired answer that are not directly relevant to the prompt. This problem adds complexity to the post-processing and performance evaluation of the model's outputs. We proposed a hypothesis that the use of more specific prompts could minimize the unnecessary sentences in the outputs. However, the results showed that our hypothesis was false. Further research is required to address this challenge and improve the model's response accuracy.

\section{Introduction}

In the era of artificial intelligence, Large Language Models (LLMs) such as GPT-4 have exhibited remarkable performance in various tasks by generating text in response to input prompts. However, one persistent problem is that these models often output irrelevant sentences along with the required response. For instance, when we input the prompt "What is 1 + 1?", the LLM often returns an additional sentence, "The answer to that question is 2," where the part "The answer to that question is" is unnecessary.

This undesirable phenomenon complicates subsequent operations such as post-processing and performance evaluation, as it can lead to false negatives where correct answers are interpreted as incorrect due to additional information. Furthermore, addressing this inconsistency in the post-processing stage often necessitates a context-specific approach, which is especially challenging as it is impossible to predict the model's output precisely.

Our hypothesis was that the occurrence of irrelevant sentences could be reduced by providing more specific prompts to the LLM. We suggested that if we changed our prompt from "What is 1 + 1?" to "Calculate: 1 + 1" or "Solve: 1 + 1", the LLM could more accurately output the direct answer. This paper details our experiment process and the consequential results, which contradicted our hypothesis. The results revealed that changing the prompt did not significantly reduce the occurrence of irrelevant sentences. 

\end{document}