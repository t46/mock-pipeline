\documentclass{article}
\usepackage[utf8]{inputenc}

\title{Improving Directness of Responses in Large Language Models}
\author{Author Name}
\date{Date}

\begin{document}

\maketitle

\begin{abstract}
Large Language Models (LLMs), such as GPT-4, have shown remarkable capabilities in generating human-like text. However, they often produce responses that include extraneous information not directly related to the input prompt. This paper investigates the problem of LLMs generating unnecessary sentences in their responses, which complicates post-processing and evaluation of the model's performance. We hypothesize that modifying the input prompt to instruct the model to provide a direct answer could potentially mitigate this issue. However, our findings indicate that this hypothesis does not hold true.
\end{abstract}

\section{Introduction}

Large Language Models (LLMs) like GPT-4 have revolutionized the field of natural language processing with their ability to generate coherent and contextually relevant text. These models take a text input, referred to as a prompt, and generate a text output in response. However, a significant challenge arises when these models generate sentences that are not directly related to the input instructions. 

For instance, when asked a simple question like "What is 1 + 1?", the LLM might respond with "The answer to that question is 2". While the response is correct, it includes an unnecessary sentence. The ideal response would be a direct answer, "2". This issue complicates the evaluation of the model's performance, especially when the evaluation involves comparing the model's output with a predefined correct answer. 

The problem is further exacerbated by the fact that it is difficult to devise a general post-processing method to handle this issue, as the extraneous text generated by the model can vary widely depending on the input prompt. 

In this study, we hypothesized that instructing the model to provide a direct answer in the input prompt could encourage the model to generate more direct responses. However, our findings indicate that this approach does not effectively address the problem. 

\end{document}